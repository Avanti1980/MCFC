\documentclass{ctexart}
\usepackage{avanti}

\begin{document}
\title{最大匹配与最小点覆盖}
\author{圆眼睛的阿凡提哥哥}
\date{\today}
\maketitle

设二部图$\Gcal = (\Vcal, \Ecal)$,其中$\Vcal = \Vcal_1 \uplus \Vcal_2$,$\Ecal \subseteq \Vcal_1 \times \Vcal_2$,$\delta(v)$为与点$v$相连的边的集合。若$\Mcal \subseteq \Ecal$且其中任意两条边没有公共顶点,即不存在长度$\ge 2$的路径,则称$\Mcal$为匹配(matching),其可表示为向量$\xv \in \Zbb_+^{|\Ecal|}$满足对任意$v \in \Vcal$有$\sum_{e \in \delta(v)} x_e \le 1$。若$\Ccal \subseteq \Vcal$使得$\Gcal$的每条边都至少有一个顶点属于$\Ccal$,则称$\Ccal$为覆盖(cover),其可表示为向量$\zv \in \Zbb_+^{|\Vcal|}$使得对任意$(u,v) \in \Ecal$有$z_u + z_v \ge 1$。

设$\Av \in \{ 0,1 \}^{|\Vcal| \times |\Ecal|}$是二部图$\Gcal$对应的关联矩阵,即$a_{v,e} = 1_{e \in \delta(v)}$,则
\begin{align*}
    \forall v \in \Vcal, \sum_{e \in \delta(v)} x_e \le 1 & \Longleftrightarrow \Av \xv \le \ev      \\
    \forall (u,v) \in \Ecal, ~ z_u + z_v \ge 1            & \Longleftrightarrow \Av^\top \zv \ge \ev
\end{align*}

\section{最大匹配}

所有匹配中,势最大的称为\blue{最大匹配},求解最大匹配可形式化成
\begin{align} \label{eq: max-matching}
    \max_{\xv} \{ \ev^\top \xv : \xv \in \Zbb_+^{|\Ecal|}, ~ \Av \xv \le \ev \}
\end{align}
由于第一个约束的存在,这是一个整数规划,难以直接求解,将可行域放松成连续域可得线性规划
\begin{align} \label{eq: relax-max-matching}
    \max_{\xv} \{ \ev^\top \xv : \xv \ge \zerov, ~ \Av \xv \le \ev \}
\end{align}
注意$\{ \xv \ge \zerov, ~ \Av \xv \le \ev \} \Longleftrightarrow [\Av; -\Iv] \xv \le [\ev; \zerov]$,由于二部图的关联矩阵必然是\href{https://avanti1980.github.io/notes-on-math/posts/matrix/TU-matrix.html}{全幺模矩阵},故$[\Av; -\Iv]$也是全幺模矩阵,又$[\ev; \zerov]$是整数向量,故凸多面体$\{ \xv \ge \zerov, ~ \Av \xv \le \ev \}$的\href{https://avanti1980.github.io/notes-on-math/posts/convex-optimization/extreme-point.html}{极点}是整数向量。由于线性规划必然在极点处取最优,因此式(\ref{eq: relax-max-matching})的最优解就是式(\ref{eq: max-matching})的最大匹配。

上述将离散整数约束替换为连续实数约束的操作,其实是将可行域由匹配集合扩大成其凸包。

\begin{theorem}
    记匹配$\Mcal$对应的表示向量为$\xv^{(\Mcal)}$,$\Pcal (\Gcal) \triangleq \conv \{ \xv^{(\Mcal_1)}, \xv^{(\Mcal_2)}, \ldots \}$,$\Qcal (\Gcal)$定义为:
    \begin{align*}
        \Qcal (\Gcal) = \{ \xv \mid \xv \ge \zerov, ~ \Av \xv \le \ev \} = \left\{ \xv \in \Rbb_+^{|\Vcal|} \mid \forall v \in \Vcal, \sum_{e \in \delta(v)} x_e \le 1 \right\}
    \end{align*}
    那么$\Pcal (\Gcal) = \Qcal (\Gcal)$。
\end{theorem}

\begin{proof}
    正向比较简单,对任意$\xv = \sum_{i \in [n]} \alpha^{(\Mcal_i)} \xv^{(\Mcal_i)} \in \Pcal(\Gcal)$,易知
    \begin{align*}
        \sum_{e \in \delta(v)} x_e & = \sum_{e \in \delta(v)} \sum_{i \in [n]} \alpha^{(\Mcal_i)} x^{(\Mcal_i)}_e = \sum_{i \in [n]} \alpha^{(\Mcal_i)} \underbrace{\sum_{e \in \delta(v)} x^{(\Mcal_i)}_e}_{\le 1} \le \sum_{i \in [n]} \alpha^{(\Mcal_i)} = 1
    \end{align*}
    其中不等号是因为对任意匹配,点$v$相连的边中最多只有一条属于该匹配。

    反向较为麻烦,对任意$\xv \in \Qcal (\Gcal)$,设$\supp(\xv) = \{ e \in \Ecal \mid x_e > 0 \}$。下面对$|\supp(\xv)|$进行归纳,若$|\supp(\xv)| = 0$,则$\xv = \zerov$就是零匹配;若$|\supp(\xv)| = 1$,显然$\xv$可以表示成零匹配和单边匹配的凸组合。若$|\supp(\xv)| \ge 2$,分两种情况讨论:
    \begin{itemize}
        \item $\supp(\xv)$不是匹配,则$\supp(\xv)$包含长度$\ge 2$的路径,不妨就设为$v_1 \xrightarrow{e_1} v_2 \xrightarrow{e_2} v_3$,由于$x_{e_1}, x_{e_2} > 0$,故$x_{e_1}, x_{e_2} < 1$,否则$\sum_{e \in \delta(v_2)} x_e = x_{e_1} + x_{e_2} > 1$。引入
              \begin{align*}
                  d_e = \begin{cases}
                      1 & e = e_1 \\ -1 & e = e_2 \\ 0 & \ow
                  \end{cases}
              \end{align*}
              现考虑$\xv + \epsilon \dv$,当$\epsilon$增大时,$x_{e_1} + \epsilon d_{e_1}$增大,$x_{e_2} + \epsilon d_{e_2}$减小,当$x_{e_2} + \epsilon d_{e_2}$变为零时,记此时的$\epsilon$为$\epsilon_1$,定义$\xv_1 \triangleq \xv + \epsilon_1 \dv$;同理对于$\xv - \epsilon \dv$,当$\epsilon$增大时,$x_{e_1} - \epsilon \dv_{e_1}$减小,$x_{e_2} - \epsilon \dv_{e_2}$增大,当$x_{e_1} - \epsilon \dv_{e_1}$变为零时,记此时的$\epsilon$为$\epsilon_2$,定义$\xv_2 \triangleq \xv - \epsilon_2 \dv$,那么
              \begin{align*}
                  \epsilon_2 \epsilon_1 \dv = \epsilon_2 \xv_1 - \epsilon_2 \xv = \epsilon_1 \xv - \epsilon_1 \xv_2 \Longrightarrow \xv = \frac{\epsilon_2}{\epsilon_1 + \epsilon_2} \xv_1 + \frac{\epsilon_1}{\epsilon_1 + \epsilon_2} \xv_2 = \conv\{ \xv_1, \xv_2 \}
              \end{align*}
              注意$|\supp(\xv_1)| = |\supp(\xv_2)| = |\supp(\xv)| - 1$,由归纳假设知$\xv_1,\xv_2 \in \Pcal(\Gcal)$,于是$\xv \in \Pcal(\Gcal)$。

        \item $\supp(\xv)$是匹配,不妨设$\supp(\xv) = \{ e_1, e_2, e_3, \ldots, e_n \}$且$x_{e_1} \le x_{e_2} \le x_{e_3} \le \cdots \le x_{e_n}$,定义
              \begin{align*}
                  \Mcal_i \triangleq \{ e_i, e_{i+1}, \ldots, e_n \}, \quad \xv^{(\Mcal_i)} = [\underbrace{0, \ldots, 0}_{1:i-1}, \underbrace{1, 1, \ldots, 1}_{i:n}, \underbrace{0, \ldots, 0}_{n+1:|\Ecal|}], \quad i \in [n]
              \end{align*}
              则
              \begin{align*}
                  \xv & = \begin{bmatrix}
                      x_{e_1} \\ x_{e_2} \\ x_{e_3} \\ \vdots \\ x_{e_n} \\ \zerov
                  \end{bmatrix} = \begin{bmatrix}
                      x_{e_1} \\ x_{e_1} \\ x_{e_1} \\ \vdots \\ x_{e_1} \\ \zerov
                  \end{bmatrix} + \begin{bmatrix}
                      0 \\ x_{e_2} - x_{e_1} \\ x_{e_2} - x_{e_1} \\ \vdots \\ x_{e_2} - x_{e_1} \\ \zerov
                  \end{bmatrix} + \begin{bmatrix}
                      0 \\ 0 \\ x_{e_3} - x_{e_2} \\ \vdots \\ x_{e_3} - x_{e_2} \\ \zerov
                  \end{bmatrix} + \cdots \\
                      & = x_{e_1} \xv^{(\Mcal_1)} + (x_{e_2} - x_{e_1}) \xv^{(\Mcal_2)} + (x_{e_3} - x_{e_2}) \xv^{(\Mcal_3)}                       \\
                      & \qquad + \cdots + (x_{e_n} - x_{e_{n-1}}) \xv^{(\Mcal_n)} + (1 - x_{e_n}) \zerov \in \Pcal (\Gcal)
              \end{align*}
    \end{itemize}
\end{proof}

由定义$\Pcal (\Gcal) = \conv \{ \xv^{(\Mcal_1)}, \xv^{(\Mcal_2)}, \ldots \}$知$\Pcal (\Gcal)$的任意极点都是$\Gcal$的匹配,反过来结论也成立。

\begin{theorem} \label{thm: extreme-point-matching}
    $\Gcal$的任意匹配都是$\Pcal$的极点。
\end{theorem}

\begin{proof}
    对任意匹配$\Mcal$和非零向量$\dv$,不妨设$d_e \neq 0$,注意$x^{(\Mcal)}_e \in \{0, 1\}$,因此$x^{(\Mcal)}_e \pm \epsilon d_e$总有一个不属于$[0,1]$,即$\xv^{(\Mcal)} \pm \epsilon \dv$总有一个不属于$\Pcal$,故$\xv^{(\Mcal)}$是$\Pcal$的极点。
\end{proof}

\section{完美匹配}

若匹配$\Mcal^\star$使得在子图$(\Vcal, \Mcal^\star)$中,所有点都有且仅有一条相连的边,则称为完美匹配(perfect matching)。完美匹配可表示为向量$\xv \in \Zbb_+^{|\Ecal|}$满足对任意$v \in \Vcal$有$\sum_{e \in \delta(v)} x_e = 1$,显然完美匹配是匹配的真子集。

\begin{theorem}
    设$\Pcal^\star (\Gcal)$为$\Gcal$的所有完美匹配构成的凸包,$\Qcal^\star (\Gcal)$定义为:
    \begin{align*}
        \Qcal^\star (\Gcal) = \{ \xv \mid \xv \ge \zerov, ~ \blue{\Av \xv = \ev} \} = \left\{ \xv \in \Rbb_+^{|\Vcal|} \mid \forall v \in \Vcal, \blue{\sum_{e \in \delta(v)} x_e = 1} \right\}
    \end{align*}
    则$\Pcal^\star (\Gcal) = \Qcal^\star (\Gcal)$。
\end{theorem}

\begin{proof}
    一方面,对任意$\xv = \sum_{i \in [n]} \alpha^{(\Mcal_i^\star)} \xv^{(\Mcal_i^\star)} \in \Pcal^\star(\Gcal)$,易知
    \begin{align*}
        \sum_{e \in \delta(v)} x_e & = \sum_{e \in \delta(v)} \sum_{i \in [n]} \alpha^{(\Mcal_i^\star)} x^{(\Mcal_i^\star)}_e = \sum_{i \in [n]} \alpha^{(\Mcal_i^\star)} \sum_{e \in \delta(v)} x_e^{(\Mcal_i^\star)} = \sum_{i \in [n]} \alpha^{(\Mcal_i^\star)} = 1 \Longrightarrow \xv \in \Qcal^\star (\Gcal)
    \end{align*}

    另一方面,对任意$\xv \in \Qcal^\star(\Gcal) \subseteq \Qcal(\Gcal) = \Pcal(\Gcal)$,设$\xv = \sum_{i \in [n]} \alpha^{(\Mcal_i)} \xv^{(\Mcal_i)}$。用反证法,若其凸组合表示中存在不完美匹配$\Mcal_j$,设$v$不是$\Mcal_j$中边的顶点,则
    \begin{align*}
        \sum_{e \in \delta(v)} x_e = \sum_{e \in \delta(v)} \sum_{i \in [n] \setminus \{j\}} \alpha^{(\Mcal_i)} x_e^{(\Mcal_i)} = \sum_{i \in [n] \setminus \{j\}} \alpha^{(\Mcal_i)} \sum_{e \in \delta(v)} x_e^{(\Mcal_i)} \le \sum_{i \in [n] \setminus \{j\}} \alpha^{(\Mcal_i)} < 1
    \end{align*}
    这和$\Qcal^\star (\Gcal)$的定义矛盾,故$\xv$的凸组合表示中不存在不完美匹配,即$\xv \in \Pcal^\star (\Gcal)$。
\end{proof}

\begin{theorem} \label{thm: extreme-point-perfect-matching}
    $\Gcal$的任意完美匹配都是$\Pcal^\star$的极点。
\end{theorem}

\begin{proof}
    完美匹配也是匹配,因此是$\Pcal$的极点,故无法由$\Pcal$中其它点的凸组合表示,又$\Pcal^\star \subseteq \Pcal$,因此也无法由$\Pcal^\star$中其它点的凸组合表示,从而也是$\Pcal^\star$的极点
\end{proof}

对于完全二部图$\Kcal_{n,n}$有$|\Ecal| = n^2$,对任意$\xv \in \Qcal^\star(\Kcal_{n,n})$有
\begin{align*}
    \xv \in \Rbb_+^{n^2}, ~ \forall v \in \Vcal, \sum_{e \in \delta(v)} x_e = 1
\end{align*}
又每个点恰有$n$条相连的边,因此$\xv$也可以写成一个$n \times n$的双随机矩阵(所有行和、列和均为$1$)。另一方面,对于完美匹配$\Mcal$,每个点有且仅有一条相连的边,其对应的$\xv^{(\Mcal)}$可以写成置换矩阵(每行、每列有且仅有一个$1$,其余为零),由定理\ref{thm: extreme-point-perfect-matching}知\blue{双随机矩阵集合的极点是置换矩阵},这就是Birkhoff-von Neumann定理。

\section{König定理}

前文已述最大匹配问题可放松成线性规划
\begin{align*}
    \max_{\xv} \{ \ev^\top \xv : \xv \ge \zerov, ~ \Av \xv \le \ev \}
\end{align*}
引入Lagrange对偶函数$\Lcal(\xv, \yv, \zv) = \ev^\top \xv + \yv^\top \xv - \zv^\top (\Av \xv - \ev)$,易知
\begin{align*}
    \frac{\partial \Lcal}{\partial \xv} = \ev + \yv - \Av^\top \zv = \zerov \Longrightarrow \Av^\top \zv - \ev = \yv \geq \zerov
\end{align*}
故对偶问题为
\begin{align} \label{eq: relax-min-vertex-cover}
    \min_{\zv} \{ \ev^\top \zv : \zv \ge \zerov, ~ \Av^\top \zv \ge \ev \}
\end{align}
显然这是将\blue{最小点覆盖}问题
\begin{align} \label{eq: min-vertex-cover}
    \min_{\zv} \{ \ev^\top \zv : \zv \in \Zbb_+^{|\Vcal|}, ~ \Av^\top \zv \ge \ev \}
\end{align}
的离散可行域放松成连续域得到的线性规划。同理由$\{ \zv \ge \zerov, ~ \Av^\top \zv \ge \ev \} \Longleftrightarrow [-\Av^\top; -\Iv] \zv \le [-\ev; \zerov]$以及$\Av$是全幺模矩阵知凸多面体$\{ \zv \mid \zv \ge \zerov, ~ \Av^\top \zv \ge \ev \}$的极点是整数向量。由于线性规划必然在极点处取最优,因此式(\ref{eq: relax-min-vertex-cover})的最优解就是式(\ref{eq: min-vertex-cover})的最小点覆盖。

综上,最大匹配、最小点覆盖这两类整数规划问题,其最优解就是将整数约束放松后导出的线性规划的最优解,且这两类相应的线性规划互为对偶问题。

\begin{theorem} [König]
    对于二部图$\Gcal = (\Vcal, \Ecal)$,设最大匹配问题的最优值为$\maxm(\Gcal)$,最小点覆盖问题的最优值为$\minvc(\Gcal)$,则有$\maxm(\Gcal) = \minvc(\Gcal)$。
\end{theorem}

\begin{proof}
    $\minvc(\Gcal) \ge \maxm(\Gcal)$是显然的,因为对最大匹配中的任意一条边,至少要覆盖其中一个顶点。

    下面证明另一个方向,若$\Ecal = \emptyset$,则$\maxm(\Gcal) = \minvc(\Gcal) = 0$,故不妨设$\Ecal$非空。对$|\Vcal|$进行归纳,若$|\Vcal| = 2$,易知$\maxm(\Gcal) = \minvc(\Gcal) = 1$。若$|\Vcal| > 2$,设$\zv^\star$是最小点覆盖问题的最优解,由于存在点$v$使得$z_v^\star > 0$,故根据互补松弛条件可得
    \begin{align*}
        z_v^\star (\Av_{v,:} \xv^\star - 1) = 0 \Longrightarrow 1 = \Av_{v,:} \xv^\star = \sum_{e \in \delta(v)} x_e^\star
    \end{align*}
    又原问题的最优解$\xv^\star$是最大匹配,故$v$出现在所有的最大匹配中,于是
    \begin{align*}
        \maxm(\Gcal \setminus \{v\}) = \maxm(\Gcal) - 1
    \end{align*}
    由归纳假设知$\maxm(\Gcal \setminus \{v\}) = \minvc(\Gcal \setminus \{v\})$,于是
    \begin{align*}
        \minvc(\Gcal) & \le \minvc(\Gcal \setminus \{v\}) + 1 \\
                      & = \maxm(\Gcal \setminus \{v\}) + 1    \\
                      & = \maxm(\Gcal)
    \end{align*}
\end{proof}

König定理还可进一步推广,设$b$-匹配对应的表示向量满足对任意$v \in \Vcal$有$\sum_{e \in \delta(v)} x_e \le b_v$;$c$-点覆盖对应的表示向量满足对任意$e = (u,v) \in \Ecal$有$z_u + z_v \ge c_e$,易知有
\begin{align*}
    \max_{\xv} \{ \cv^\top \xv : \xv \ge \zerov, ~ \Av \xv \le \bv \} = \min_{\zv} \{ \bv^\top \zv : \zv \ge \zerov, ~ \Av^\top \zv \ge \cv \}
\end{align*}
即最大$c$-加权$b$-匹配等于最小$b$-加权$c$-点覆盖。

\section{最大流与最小割}

类似于最大匹配和最小点覆盖,最大流和最小割也是一组对偶问题。给定有向流网络$\Gcal = (\Vcal, \Ecal)$、源点$s$、汇点$t$,设$\delta_{\text{in}}(v)$是以点$v$为终点的入边集合、$\delta_{\text{out}}(v)$是以点$v$为起点的出边集合,$\Av \in \{ 0, \pm 1 \}^{|\Vcal| \times |\Ecal|}$是$\Gcal$对应的关联矩阵,即
\begin{align*}
    a_{v,e} = \begin{cases}
        1  & e \in \delta_{\text{in}} (v)  \\
        -1 & e \in \delta_{\text{out}} (v) \\
        0  & \ow
    \end{cases}
\end{align*}
$\Av_{\overline{st}}$为$\Av$去掉$s$、$t$对应行的子矩阵,注意有向流网络中源点$s$只有出边、汇点$t$只有入边,因此$\Av_{\overline{st}}$其实也是$\Gcal$删除$s$、$t$及其所有相连边后的有向图的关联矩阵,故$\Av_{\overline{st}}$是全幺模矩阵。

最大流问题可形式化为线性规划:
\begin{align*}
    \max_{\xv} \{ \Av_t \xv : \zerov \le \xv \le \cv, ~ \Av_{\overline{st}} \xv = \zerov \}
\end{align*}
其中$\Av_t$是$\Av$中汇点$t$对应的行,$\zerov \le \xv \le \cv$约束流的上下界,$\Av_{\overline{st}} \xv = \zerov$约束非源点、汇点的流量要守恒。注意
\begin{align*}
    \{ \xv \mid \zerov \le \xv \le \cv, ~ \Av_{\overline{st}} \xv = \zerov \} \Longleftrightarrow [\Av_{\overline{st}}; -\Av_{\overline{st}}; \Iv; -\Iv] \xv \leq [\zerov; \zerov; \cv; \zerov]
\end{align*}
由$\Av_{\overline{st}}$是全幺模矩阵知$[\Av_{\overline{st}}; -\Av_{\overline{st}}; \Iv; -\Iv]$也是全幺模矩阵,若流量上限$\cv$是整数向量,则可行域$\{ \zv \mid \zerov \le \xv \le \cv, ~ \Av_{\overline{st}} \xv = \zerov \}$的极点也是整数向量,即最大流是整数流。

引入Lagrange对偶函数$\Lcal(\xv, \yv, \zv, \wv) = \Av_t \xv + \yv^\top \xv - \zv^\top (\xv - \cv) - \wv_{\overline{st}}^\top \Av_{\overline{st}} \xv$,易知
\begin{align*}
    \frac{\partial \Lcal}{\partial \xv} = \Av_t^\top + \yv - \zv - \Av_{\overline{st}}^\top \wv_{\overline{st}} = \zerov \Longrightarrow \Av_{\overline{st}}^\top \wv_{\overline{st}} + \zv \ge \Av_t^\top
\end{align*}
故对偶问题为
\begin{align*}
    \min_{\wv_{\overline{st}}, \zv} \{ \cv^\top \zv : \zv \ge \zerov, ~ \Av_{\overline{st}}^\top \wv_{\overline{st}} + \zv \ge \Av_t^\top \}
\end{align*}
注意
\begin{align*}
    \{ \zv \mid \zv \ge \zerov, ~ \Av_{\overline{st}}^\top \wv_{\overline{st}} + \zv \ge \Av_t^\top \} \Longleftrightarrow [-\Av_{\overline{st}}^\top, -\Iv; \zerov, -\Iv] [\wv_{\overline{st}}; \zv] \leq [-\Av_t^\top; \zerov]
\end{align*}
由$\Av_{\overline{st}}$是全幺模矩阵知$[-\Av_{\overline{st}}^\top, -\Iv; \zerov, -\Iv]$也是全幺模矩阵,故对偶问题的最优解$\wv_{\overline{st}}^\star$、$\zv^\star$也是整数向量。

$\wv_{\overline{st}}^\star$的维度为$|\Vcal| - 2$,与$\Av_{\overline{st}}$的行对应,现添加$w_s^\star = 0$、$w_t^\star = -1$将其扩充为$\wv^\star$,与$\Av$的行对应,于是$\Av^\top \wv^\star + \zv^\star = \Av_{\overline{st}}^\top \wv_{\overline{st}}^\star - \Av_t^\top + \zv^\star \ge \zerov$。由于$\cv$非负,故$\zv^\star$应尽量的小,从而$\zv^\star = \max \{ \zerov, - \Av^\top \wv^\star \}$,即对$e = (u,v) \in \Ecal$有$z^\star_e = \max \{ 0, w_u^\star - w_v^\star \}$。

定义$\Scal = \{ v \in \Vcal \mid w_v^\star \ge 0 \}$,$\overline{\Scal} = \Vcal \setminus \Scal$,$\delta_{\Scal, \overline{\Scal}} \triangleq \{ (u,v) \in \Ecal \mid u \in \Scal, ~ v \in \overline{\Scal} \}$为所有起点属于$\Scal$、终点属于$\overline{\Scal}$的边的集合。显然$s \in \Scal$、$t \in \overline{\Scal}$,在将所有$\delta_{\Scal, \overline{\Scal}}$中的边删除后,$s$、$t$不再连通,因此$\delta_{\Scal, \overline{\Scal}}$称为割(cut)。

由于$w_v^\star$都是整数,因此对任意$e = (u,v) \in \delta_{\Scal, \overline{\Scal}}$有$z_e^\star \ge w_u^\star - w_v^\star \ge 1$,因此
\begin{align*}
    \cv^\top \zv^\star \ge \sum_{e \in \delta_{\Scal, \overline{\Scal}}} c_e z_e^\star \ge \sum_{e \in \delta_{\Scal, \overline{\Scal}}} c_e \ge \sum_{e \in \delta_{\Scal, \overline{\Scal}}} x_e^\star \ge \sum_{e \in \delta_{\text{in}}(t)} x_e^\star = \Av_t \xv^\star \ge \cv^\top \zv^\star
\end{align*}
其中第一个不等号是因为$z_e^\star \ge 0$;第二个不等号是因为对任意$e \in \delta_{\Scal, \overline{\Scal}}$有$z_e^\star \ge 1$;第三个不等号是因为$c_e$是边$e$的流量上限;第四个不等号是因为从$\Scal$到$\overline{\Scal}$的流量未必会全部进入汇点,可能会有一部分通过从$\overline{\Scal}$到$\Scal$的边再折回$\Scal$;第五个不等号是因为弱对偶性。

综上所有的不等号都取等号,由此可以得到一些有趣的结论:
\begin{itemize}
    \item 根据第一个不等号取等号,对任意$e \not \in \delta_{\Scal, \overline{\Scal}}$有$z_e^\star = 0$,即对任意$\Scal$内部的边$e$、$\overline{\Scal}$内部的边$e$、$\delta_{\overline{\Scal}, \Scal}$中的边$e$,都有$z_e^\star = 0$;
    \item 根据第二个不等号取等号,对任意$e = (u,v) \in \delta_{\Scal, \overline{\Scal}}$有$z_e^\star = 1$,故只可能是$w_u^\star = 0$、$w_v^\star = -1$,于是对任意$\Scal$内部的边$e = (p, u)$,必然有$w_p^\star = 0$,否则$z_e^\star \ge w_p^\star - w_u^\star > 0$,与前一个结论矛盾,依此类推,对所有$\Scal$中的点$u$都有$w_u^\star = 0$。同理,对所有$\overline{\Scal}$中的点$v$都有$w_v^\star = -1$;
    \item 根据第三个不等号取等号,当流量达到最大时,$\delta_{\Scal, \overline{\Scal}}$中的每条边上的流量都达到上限,这个也可由互补松弛条件$z_e (x_e - c_e) = 0$得到:$z_e^\star = 1 > 0 \Longrightarrow x_e^\star = c_e$;
    \item 根据第四个不等号取等号,从$\Scal$到$\overline{\Scal}$的流量全部进入$t$,不存在折回$\Scal$的情况,即$\delta_{\overline{\Scal}, \Scal}$中的每条边上的流量都是零,这个也可由互补松弛条件$y_e x_e = 0$得到:注意此时$z_e^\star = 0 > -1 = w_u^\star - w_v^\star$,故$y_e^\star = z_e^\star - (w_u^\star - w_v^\star) > 0$,从而$x_e^\star = 0$。
\end{itemize}



\end{document}

